\chapter{Virtual Knowledge Graphs}
\label{cha:vkg}

\section{Cos'è un Virtual Knowledge Graph}
\label{sec:vkg_description}
Invenzione VKG da parte di Google

Vantaggi di un VGK

Struttura di un vkg: ontology, mapping, schema

\section{Il Virtual Knowledge Graph system Ontop}
\label{sec:vkg_ontop}
VKG system open source che segue gli standard W3C 

\subsection{Intermediate Query language}
Ontop era inizialmente basato su Datalog poi passato ad un proprio linguaggio interno (Intermediate Query)

IQ: rappresentazione usata per tradurre le query degli utenti in SPARQL nelle query SQL dei mapping

\subsection{Esempi di utilizzo}
Esempi di utilizzo di Ontop in ambienti aziendali e accademici


