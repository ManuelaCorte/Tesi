\chapter{Esperienza}
\label{cha:experience}

\section{Il modulo bi-connector}
\label{sec:bi-connector}
Bi-connector: progetto con lo scopo di poter usare strumenti di BI come Tableau su ontologie. (a differenza di altre soluzioni NoSQL come MongoDB non esistono connettori pre-forniti)

\subsection{Creazione database}
\label{sec:bi-connector_db}
Dai file dell'ontologia (.ttl) vengono estratte delle viste e con queste viste viene creato in locale un database PostgreSQL. (Per visualizzarlo ho usato DBeaver con la classe ProfJDBC.java)

\subsection{Parsing query SQL}
\label{sec:bi-connector_parsing}
Su questo database è possibile fare query SQL. 
Viene fatto il parsing di questa query e viene tradotta in un albero di nodi IQ che viene usato dall'ontologia per rispondere alla query.

\section{Stato iniziale bi-connector}
\label{sec:experience_start}
Inizialmente veniva utilizzato il parser SQL interno di Ontop che però era molto limitato essendo usato per le query usate nei mapping che sono tipicamente semplici (union of conjunctive queries).
Si è poi passato a JSqlParser (libreria esterna) che trasforma la query in una gerarchia traversabile di classi Java.
Avendo un insieme di query limitato che era riconosciuto molto delle query erano o riscritte in una forma semplificata o non supportate del tutto

\section{Analisi prerequisiti}
\label{sec:section2}
Analisi di quali fossero i costrutti usati nelle query automaticamente generate da Tableau e analisi di quale fosse il comportamento specifico di PostgreSQL su queste keyword

