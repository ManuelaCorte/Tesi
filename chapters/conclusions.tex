\chapter{Conclusioni}
\label{cha:conclusions}
Il tirocinio presso Ontopic mi ha permesso di acquisire nuove competenze sia da un punto di vista puramente tecnico che 
personale. Infatti, per la prima volta ho avuto la possibilità di lavorare su del codice a livello aziendale dove 
affidabilità e leggibilità del codice implementato sono gli aspetti più importanti e, da un punto di vista più
pratico approfondire le mie conoscenze in linguaggi quali Java e SQL. Inoltre ho potuto sperimentare in prima persona cosa significhi lavorare in un team di sviluppatori e come in esso 
vengano divise le responsabilità.

Un altro aspetto fondamentale che ha certamente contribuito all'esito positivo di quest'esperienza è sicuramente il 
mio interesse personale verso il mondo della data integration e dell'analisi dati. Durante questo 
tirocinio ho avuto infatti la possibilità di interfacciarmi con moltissime tecnologie in questo campo come i Virtual Knowledge
Graph e tutti gli standard legati al semantic web (RDF, R2RML, SPARQL, \dots) ad essi associati.

\section{Possibili sviluppi futuri}
\label{sec:conclusions_future}
Il modulo bi-connector al quale ho lavorato durante tirocinio ha ampie potenzialità di crescita. Infatti, nonostante l'estensione
del parser SQL effettuata ancora moltissime funzionalità di SQL non sono implementate. Questo è particolarmente vero per quanto 
riguarda le funzioni; infatti SQL presenta moltissime funzioni pre implementate utili che non sono però supportate nel sistema
come ad esempio buona parte delle funzioni per la manipolazione di date e fusi orari.

Inoltre è possibile includere il supporto per altri strumenti di Business Intelligence come ad esempio PowerBI o Qlik. Infatti 
a seguito della fine del mio tirocinio si è già iniziato a lavorare per estendere il supporto a PowerBI.

Supporto ODBC oltre che JDBC?
